%% joser_template.tex
%% V1
%% 2008/12/15
%% by Davide Brugali
%% This is a skeleton file demonstrating the use of joser1.cls
%% with a JOSER paper
%%
%% This file is a modified version of bare_jrnl_compsoc.tex V1.3
%% by Michael Shell for IEEE CS journal papers
%% http://www.michaelshell.org/
%%
%%*************************************************************************
%% Legal Notice:
%% This code is offered as-is without any warranty either expressed or
%% implied; without even the implied warranty of MERCHANTABILITY or
%% FITNESS FOR A PARTICULAR PURPOSE!
%% User assumes all risk.
%% In no event shall JOSER or any contributor to this code be liable for
%% any damages or losses, including, but not limited to, incidental,
%% consequential, or any other damages, resulting from the use or misuse
%% of any information contained here.
%%
%% This work is distributed under the LaTeX Project Public License (LPPL)
%% ( http://www.latex-project.org/ ) version 1.3, and may be freely used,
%% distributed and modified. A copy of the LPPL, version 1.3, is included
%% in the base LaTeX documentation of all distributions of LaTeX released
%% 2003/12/01 or later.
%% Retain all contribution notices and credits.
%% ** Modified files should be clearly indicated as such, including  **
%% ** renaming them and changing author support contact information. **
%%
%%*************************************************************************

% Fixes by Dave
\setlength{\paperheight}{11in}
\PassOptionsToPackage{pdfpagelabels=false}{hyperref} 



\documentclass[10pt,journal,compsoc]{joser1}

% *** GRAPHICS RELATED PACKAGES ***
%
\ifCLASSINFOpdf
   \usepackage[pdftex]{graphicx}
\else
   \usepackage[dvips]{graphicx}
\fi

%%%%%%%%%%%%%%%%%%%%%%%%%%%%%%%%%%%%%%%%%%%%%%%%%%%%%%%%%%%%%%%%%%%%%%%
%%%%%%%%%%%%%%%%%%%%%%% will be inserted by the editor %%%%%%%%%%%%%%%%
%%%%%%%%%%%%%%%%%%%%%%%%%%%%%%%%%%%%%%%%%%%%%%%%%%%%%%%%%%%%%%%%%%%%%%%
\journalnumber{1}                       %will be inserted by the editor
\journalvolume{1}                       %will be inserted by the editor
\journalmonth{September}                %will be inserted by the editor
\journalyear{2013}                      %will be inserted by the editor
\articlefirstpage{1}                  %will be inserted by the editor
\articlelastpage{10}                   %will be inserted by the editor
\setcounter{page}{1}                  %will be inserted by the editor
%%%%%%%%%%%%%%%%%%%%%%%%%%%%%%%%%%%%%%%%%%%%%%%%%%%%%%%%%%%%%%%%%%%%%%%

\copyrightauthor{D. Coleman, N. Correll}
\headoddname{F. A. AUTHOR et al./ Preparation of Papers for {\sl Journal of Software Engineering for Robotics}}%

% correct bad hyphenation here
\hyphenation{op-tical net-works semi-conduc-tor}


\begin{document}
% paper title
\title{Reducing the Barrier to Entry of \\\vskip 0.3\baselineskip Complex Robotic Motion Planning Software}

\author{
David COLEMAN$^{1}$
\qquad
Nikolaus CORRELL$^{1}$
%\qquad
%Third AUTHOR$^2$

%%%%%%%%%%%%%%%%%%%%%%%%%%%%%%%%%%%%%%%%%%%%%%%%%%%%%%%%%%%%%%%%%%%%%%%
%%%%%%%%%%%%%%%%%%%%%%% will be inserted by the editor %%%%%%%%%%%%%%%%
%%%%%%%%%%%%%%%%%%%%%%%%%%%%%%%%%%%%%%%%%%%%%%%%%%%%%%%%%%%%%%%%%%%%%%%
\thanks{{\bf Regular paper} -- Manuscript received April 19, 2009;
revised July 11, 2009.}
%%%%%%%%%%%%%%%%%%%%%%%%%%%%%%%%%%%%%%%%%%%%%%%%%%%%%%%%%%%%%%%%%%%%%%%


\IEEEcompsocitemizethanks{\IEEEcompsocthanksitem This work was
supported by xxxxxxxx (No.xxxxxxxx) (sponsor and financial support
acknowledgment goes here).\protect\\

\IEEEcompsocthanksitem Authors retain copyright to their papers
and grant JOSER unlimited rights to publish the paper
electronically and in hard copy. Use
of the article is permitted as long as the author(s) and the journal are properly
acknowledged.}

} % end author

\address{
$^1$ Department of Computer Science, University of Colorado at Boulder,
430 UCB, Boulder, Colorado 80309-0430\\
%$^2$ Department of Computing and Electronic Systems, University of
%Essex Colchester CO43SQ, UK 
}


% The paper headers
\markboth

\IEEEcompsoctitleabstractindextext{%
\begin{abstract}
As the capabilities and power of robotic motion planning software increases, the complexity and learning curve to new users also increases. 

Robotic motion planning involves a complex set of algorithms that requires many disparate fields of software engineering and robotic theory be synthesized together. Developing robot-agnostic software is another challenge that necessitates accounting for large variability in hardware designs and control paradigms. However, developing good software for these two requirements still does not help end users unless they can figure out how to use it. As robotics software gets more complex and powerful, it becomes increasingly difficult for users to get started with and customize the provided components as necessary for their particular application. In this paper a case study is presented for some of the best practices found for lowering the barrier of entry of the MoveIt motion planning software that allows users to 1) quickly get basic motion planning functionality right out of the box, 2) automate the configuration and optimization of the framework, 3) easily customize aspects of the toolchain, and 4) benchmark 
the results of different configurations.

\end{abstract}

\begin{IEEEkeywords}
Robotic Motion Planning, Frameworks, Barrier to Entry, Usability, MoveIt
\end{IEEEkeywords}}



% make the title area
\maketitle


\section{Introduction}
% The very first letter is a 2 line initial drop letter followed
% by the rest of the first word in caps.
\IEEEPARstart{M}{otion} {planning is a growing field of robotics that enables robots to move within their environment from one task or configuration to another. Its typical use case is the control of robotic arms from one location to another while taking into account the arm's reachability constraints, performing collision checking, and other constraints such as motion dynamics. 

The development of a motion planning framework involves combining many different fields of robotics and software engineering. Of particular importance is creating the structures and classes to share common data between the many different components. These basic data components include a model of a robot with collision bodies, a method for maintaining the state of the robot during planning and execution, and a method for maintaining the environment as perceived by the robot's sensors, henceforth referred to as the ''planning scene''. 

In addition to these basic common data structures, a motion planning framework requires many different functional components. Of primary importance is the motion planning module, which includes one or more algorithms suited for the solving the expected motion planning problems a robot might encounter. The field of motion planning algorithms is large and no one-size fits all solution exists yet, so a framework that is robot agnostic should likely include an assortment of algorithms. For more information on the selection of the proper algorithm for any particular planning problem, the reader is referred to [XX].

Other primary functional components include the collision checking module, which should be as fast as possible for the geometric primitives and meshes in the planning scene and robot model. A forward kinematics solver is required to propagate the robot's geometry based on its joint positions and an inverse kinematics solver is required when planning in the configuration of an end effector. Also required is a component for taking into account other potential constraints, such as joint/velocity/torque limits, and stability requirements. 

Secondary components must also be integrated into a powerful motion planning framework. Depending on what configuration space a problem was solved in, a generated motion planning solution of position waypoints must be parameterized into a time-variant trajectory to be executed. A controller manager must decide the proper low level controllers for the necessary joints for each trajectory. Finally a perception interface must update the planning scene with recognized objects from a perception pipeline as well as optional raw sensor data.

Higher level applications are then built on top of these motion planning components to coordinate more complex tasks, such as pick and place routines. Other optional components of a motion planning framework can include benchmarking tools, introspection and debug tools, as well as user-facing graphical user interfaces. 


\subsection{Barriers to Entry}

The aforementioned components of a motion planning framework all require varying degrees of customization and optimization for any particular robot. Choosing the right parameters for each utilized algorithm and framework component typically involves expert human input using domain-specific knowledge. As the code base for a framework becomes more powerful, so too does its size and complexity. The ability for a novice user to have the breadth of knowledge to customize every aspect of the planning tool chain becomes insurmountable for most. Therefore, one of the emerging requirements of a robot agnostic motion planning framework is implementing mechanisms that will automatically setup and tune the pipeline for arbitrary robots.

Talk about paradox of active user?

Discuss methods to lower barrier of entry:

Very easy to run / quick to setup demo (Hello World)
User interface with well laid out user affordances
Automatic generation of configuration files
Reduction of Magic Numbers
Auto-tuning of necessary parameters via algorithms
Customization of specific components of the framework
Benchmarking for quantitative results

\subsection{Benefits of Larger User Base}

This need to lower the barrier of entry is beneficial to the software itself in that it enables more users to utilize the framework. If the software is being sold for profit, the benefits of larger user base is obvious. If instead the software is a free open-source project, as most successful robotic frameworks have been [2], lowering the barrier to entry is also beneficial in that it creates a larger community of users. As the number of users increases, the speed in which bugs are identified and fixed increases. It is also hoped that development contributions to the code base increases, though this correlation is not as strong [3]. Additionally, one of the key strengths of a larger community for an open source project is increased participation of users assisting with quality assurance, documentation, and support [4].

Another benefit of lowering the barrier of entry is its implications in educational and research applications?

\section{Related Work}

In this paper we will be using the new MoveIt Motion Planning Framework as our case study for barriers of entry to the usage of robotics software. There already exists a number of motion planning frameworks available today, both open and closed source. A quick survey of these software projects...

OpenRave
OOPSMP
MPK - Motion Planning Kit
Motion Strategy Library
MoveIt [1]

\section{MoveIt Motion Planning Framework}

Details about Moveit

ROS [cite] sdas

asdasdas asd asd asdasdasd asd as

\section{Lowering the Barrier of Entry}

The following features have been implemented in MoveIt with the motivation to attract as many users as possible. The results of these features on the size of the user base will be discussed in a later section.

A. Basic Motion Planning Out of the Box

1. MoveIt Setup Assistant

One of the main features of MoveIt that has made it popular is its ratio of power and features to required setup time. A beginner to motion planning software can with very little effort take a kinematic model of an arbitrary articulated rigid body robot and execute motion plans in a virtual environment. With a few extra steps of setting up the correct hardware interfaces, one can then execute the motion plans on their actual robotic hardware.

The main facility that provides out of the box support for beginners is the MoveIt Setup Assistant (SA). The SA is a graphical user interface that steps new users though the initial configuration requirements of using a custom robot with the motion planning framework (Figure 1). It accomplishes the task of automatically generating text-based configuration files necessary for the initial operation of MoveIt. These configurations include a self-collision matrix, planning group definitions, robot poses, end effector semantics, virtual joints, and passive joints. A three dimensional model of the robot being configured is displayed on the right side of the SA GUI and various links on the robot are highlighted during configuration to visually confirm the actions of the user.

FIGURE 1: Screenshot of MoveIt Setup Assistant

The required input for the SA is a robot model file, which currently accepts the Universal Robotic Description Format (URDF) [5] or Collada [6] file formats. These XML schemas describe the physical layout of a robot's link, joints, and other necessary modeling components. They can also combine appropriate 3D CAD meshes, collision models, dynamics, joint limits, sensors and other components. Using a properly formatted robot model file, MoveIt can automatically accomplish many of the required tasks in a motion planning framework including forward and inverse kinematics, collision checking, and joint limit enforcement.

If one truly desired, the steps within the SA could almost entirely be automated themselves, but were kept manual so as to allow edge cases and unusual customizations to be accomplished.

2. MoveIt Rviz Motion Planning Plugin

The details of the automated configuration is left for the next section, but after the steps in the SA are completed the output is a ROS package containing a collection of configuration files and launch scripts (roslaunch files). The launch files include a demo script that will startup a visualization tool (rviz) with the new robot loaded and ready to run state of the art motion planning algorithms in non-physics based simulation. Using simple mouse-based interactive markers and the robot's various planning groups, the user can configure simple motion planning problems and quickly test the framework's capabilities.  

An example demo task would be using the mouse to drag a robot end effector from a start position to a goal position around some virtual obstacle. Using another GUI - the MoveIt Rviz Motion Planning Plugin - the user can click the ?Plan? button and watch MoveIt effortlessly plan the arm in a collision free path around the obstacle (Figure 2).

FIGURE 2: Screenshot of MoveIt Rviz Motion Planning Plugin

It is important to emphasize the effect of a quick ?Getting Started? demo on a new user unaccustomed to MoveIt or motion planning frameworks in general. The reinforcement effect of initial success encourages the novice and enables them to start going deeper into the functionality and code base. If the entry barrier is too low, a new user will likely give up and turn to other frameworks or custom solutions rather than continue to blindly fix software that they have no experience in.

3. Hardware Configuration and Execution

Once the user is comfortable with the basic tools and features provided by MoveIt, the next step is to configure their robot's various actuators and control interfaces to accept trajectory commands from MoveIt. This step usually requires some custom coding to account for the specifics of the robot hardware - the communication bus, real-time requirements, and control theory implementations. At the abstract level, all MoveIt requires is that the robot hardware accepts a standard ROS trajectory message containing a discretized set of time-variant waypoints including desired positions, velocities, and accelerations. 

B. Automate the configuration and optimization of the framework

The size and complexity of a feature-rich motion planning framework like MoveIt requires many parameters and configurations of the software be automatically setup and tuned. MoveIt accomplishes this both in the setup phase of a new robot - using the Setup Assistant - and sometimes during the runtime of the application.

1. Self Collision Matrix

The second step of the SA is the generation of a self-collision matrix for the robot. This collision matrix encodes pairs of links on a robot that never need to be checked for self-collision due to the kinematic infeasibility of there actually being a collision. Reasons for having collision checking disabled between two links includes 1) links that can never reach each other kinematically, 2) adjoining links that are connected and so are by design in collision, and 3) links that are always in collision for any other reason including inaccuracies in the robot model and precision error. This self-collision matrix is generated by running the robot through tens of thousands of random joint configurations and recording statistics of each link's collision status. The algorithm then automatically populates a list of link pairs that have been determined to never need to be collision checked. This saves future motion planning runs time because it reduces the amount of collision checks that are required.

2. Semantic Robotic Description Format

The other six steps of the SA all provide graphical front ends for the data required to populate the semantic robotic description format (SRDF) file used by MoveIt. The SRDF provides meta data of the robot model useful to motion planning, such as which set of joints constitutes an arm and which set of links is considered part of the end effector. Requiring the user to configure all the required semantic information by hand in a text editor would be far more tedious and difficult than using an interface that populates the available options for each required field.

The last step of the SA is to generate all launch scripts and configuration files. Not only does this step involve outputting to file the collected configurations during the step-by-step user interface, but it all generates a series of default configuration and launch files that are customized for the particular robot using the URDF and SRDF information. These defaults include the velocity and acceleration limits for each joint, the kinematic solvers for each planning group, the available planning algorithms and projection evaluators for planning. Default planning adapters are setup for pre and post-processing of motion plans, such as fixing slightly invalid start states and smoothing generated trajectories. Default benchmarking setups, controller and sensor manager scripts, and empty object databases are all generated. 

3. Internal Optimization

I think there are examples of this but Ioan knows more about this than I...

C. Easily customize aspects of the toolchain

Out of the box MoveIt lowers the barrier to entry by not requiring the user to provide their own implementation of any of the components in the motion planning framework. The Open Motion Planning Library (OMPL) [7] is configured as the default set of utilized planning algorithms. The Fast Collision Library (FCL) [8] is pre-configured as the collision checking component and the Orocos Kinematics and Dynamics Library (KDL) [9] is the default kinematics solver for kinematic chains. By default no perception components are configured

All these default choices however are limiting to more advanced users who have their own research or application-specific needs to fulfil. All of the just mentioned components are setup using plugin interfaces that are shared objects loaded at run time. Using common plugin interfaces, MoveIt can easily be customized by user created plugins for any or all of the motion planning components.

MoveIt is a plugin-centric framework that mostly avoids using message-passing inter-component communication, such as ROS messages. This decreases much of the latency delays inherent in message passing techniques and increases. The extensibility of the framework is greatly enhanced by not forcing users to use any particular algorithmic approach. Essentially, MoveIt provides a set of data sharing and synchronization tools, sharing between all components the robot's model and state.

One plugin in particular that MoveIt reduces the barrier to entry for customization is the inverse kinematics plugin. The default KDL plugin uses numerical techniques to convert from an end effector to joint configuration space. A must faster solution can be achieved by configuring OpenRave's IKFast [10] plugin that analytically solves the inverse kinematics problem. A combination of MoveIt scripts and the IKFast Robot Kinematics Compiler automatically generates the C++ code and plugin needed to increase the speed of motion planning solutions by up to 3 orders of magnitude.

D. Benchmark the results of different configurations

Be able to configure and switch out motion planning algorithms and approaches is a powerful feature of MoveIt, but its usefulness is limited without the ability to quantify the results of any changes. Optimization criteria such as path length, planning time, smoothness, distance to nearest obstacle, and energy minimization need benchmarking tools to enable users and developers to find the correct set parameters and components for any given robotic application.

MoveIt again provides a low barrier to entry for benchmarking with easy to create benchmarking configuration files that allow each test to be setup for comparison against other algorithms and values. Multivariable parameter sweeps for finding the optimal value for an algorithm's performance can be accomplished by simply supplying a upper and lower search value as well as an increment amount. Results can be output into generic formats for use in different plotting tools.

\section{Results}

Quantify MoveIt's popularity so far
Released 05/06/2013
Number of debian downloads? ask Tully to get this?
163 users on the MoveIt mailing list
Show plot of MoveIt posts over time
Show plot of users on mailing list over time
Show plots from https://www.ohloh.net/p/moveit

\section{Usability Issues with MoveIt}

There are still barriers to entry
Setting up controllers is difficult
Large code base is intimidating
Built by one programmer so not the easiest layout


\section{Conclusion}
Beyond the usual considerations in building a successful motion planning framework for robotics, an open source project that desires to maintain an active user base needs to take into account the barrier of entry to new users. As the algorithms become more complicated and the number of components and code base increases, configuring an arbitrary robot to utilize this framework becomes a daunting task requiring domain-specific expertise in a very large breadth of theory and implementation. To account for this, quick and easy initial configuration, with partially automated optimization, and easily extensible components for future customization are becoming a greater necessity in motion planning and robotic software engineering in general. 

\section*{Acknowledgments}
The authors would like to thank...

Todo remove these ~\cite{IJSEK1996:Stewart, SEER2007:Vaughan, ROBIO2006:Friedmann,
SIMPAR2008:Petters} and ~\cite{JARS2006:Colon, IJAR2001:Zielinski,
IEEE-TSE1997:Stewart, ICIAS2008:Spexard,
IROS2003:Montemerlo}), short



% references section
\bibliographystyle{IEEEtran}
% argument is your BibTeX string definitions and bibliography database(s)
\bibliography{IEEEabrv,moveit_setup_bibliography}


% biography section
\begin{IEEEbiography}[{coleman}]{First Author}
received his B.\,Sc. and
M.\,Sc. degrees in mechanical engineering from the *** University,
in 1977 and 1984, respectively, and the Ph.\,D. degree in
computing from *** University, in 1992. In 1994, he was a
faculty member at *** University and in 1996 at ***
University. Currently, he is a professor in the Department of
Information System Engineering at *** University.
He has published about 100 refereed journal and conference papers.
His research interest covers robotics, software engineering, and distributed systems.
Prof. Author received research award from Science Foundation, and
the Best Paper Award of the XX International Conference in 2000 and
2006, respectively. He is a member of ACM and IEEE.
\end{IEEEbiography}

\begin{IEEEbiography}[{nikolauscorrell}]{Second Author}
received his B.\,Sc. and
M.\,Sc. degrees in mechanical engineering from the *** University,
in 1977 and 1984, respectively, and the Ph.\,D. degree in
computing from *** University, in 1992. In 1994, he was a
faculty member at *** University and in 1996 at ***
University. Currently, he is a professor in the Department of
Information System Engineering at *** University.
He has published about 100 refereed journal and conference papers.
His research interest covers robotics, software engineering, and distributed systems.
Prof. Author received research award from Science Foundation, and
the Best Paper Award of the XX International Conference in 2000 and
2006, respectively. He is a member of ACM and IEEE.
\end{IEEEbiography}


% insert where needed to balance the two columns on the last page with
% biographies

% You can push biographies down or up by placing
% a \vfill before or after them. The appropriate
% use of \vfill depends on what kind of text is
% on the last page and whether or not the columns
% are being equalized.

\vfill

% Can be used to pull up biographies so that the bottom of the last one
% is flush with the other column.
%\enlargethispage{-5in}

% that's all folks
\end{document}
